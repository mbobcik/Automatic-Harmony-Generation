\chapter{Úvod}
\label{uvod}

V dávných dobách byla hudba pouze monofonní, což znamená, 
že se skládala ze samostatných melodií bez doprovodu.
Až v průběhu středověku byla vynalezena harmonie.
Tím rozumíme skupinu tónů, které zní všechny ve stejnou dobu.
Použitím harmonie získáme bohatší a barevnější hudbu.
Rozšiřování melodií o akordy, souzvuky, je umělecký proces,
a po muzikantovi či skladateli vyžaduje znalost přechodů akordů a harmonie tónů.
Nabytí těchto znalostí je jedna věc, 
avšak jejich uvedení do praxe může být pro laika, 
nebo hudebníka bez dostatečných zkušeností velmi obtížné.
\par

Tématem této práce je vytvořit nástroj, 
který k známé melodii doplní vhodný harmonický doprovod.
Tento nástroj pak může být užitečný právě pro hudební začátečníky,
kteří si nejsou svými schopnostmi jistí. 
Dále může být využit v procesu skládání plnohodnotné hudby počítačem.
V takovém případě by navrhovaný systém následoval za generátorem melodie,
a jeho výsledek doplňoval.
Skládání hudby, tedy i harmonie, je umění a vyžaduje hudební cit,
proto podobné systémy bývají zatíženy určitou chybou, 
popřípadě může být jejich výtvor poněkud nudný oproti profesionálnímu složení.
\par

Cílem práce je seznámit se s problematikou automatické harmonizace na základě znalosti melodie,
s principem a využitím strojového učení v této úloze, a to zejména s neuronovými sítěmi.
Dalším bodem je navržení systému, který automatické generování harmonie umožní.
Poté implementovat daný systém s pomocí vhodného frameworku pro neuronové sítě.
Implementovaný systém je potřeba otestovat na vhodné datové sadě. 
Ta bude obsahovat přepis rozličných písniček ze serveru \url{www.hooktheory.com}
do formátu MIDI.
Na závěr budou diskutovány výsledky testování a možné pokračování této práce.

%úvod práce
%úvod do problému
\section{Struktura práce}
%popis struktury práce
V první kapitole této práce nalezneme přehled hudební teorie.
Jsou představeny základní kameny hudby -- tóny, jejich soustavy a alterace.
Následuje část věnovaná notopisu,
tedy přesnému zápisu hudby speciálními znaky.
\par
Druhá kapitola obsahuje popis jiných, předchozích přístupů jiných autorů k této problematice.
Mezi nimi nalezneme například model založený na shodě šablon, 
jenž porovnává části vstupní melodie s těmi které zná.
Další přístup je založen na skrytých markovových modelech, 
které odhadují nejpravděpodobnější posloupnost akordů k dané melodii.
Některé z těchto přístupů slouží jako inspirace této práce.
\par

Následuje teoretická rešerše strojového učení,
tedy disciplíny na níž stojí správnost automatizace.
Nejprve je představeno strojové učení jako takové,
potom je popis zaměřen na ty oblasti, 
které jsou v této práci využity.
Postupně jsou rozebrány neuronové sítě a neurony,
rekurentní neuronové sítě a jejich LSTM architektura.
\par

Třetí kapitola se věnuje návrhu systému,
který umožňuje automatické generování harmonie na základě vstupní melodie.
\todo{DOPSAT podle diplomky}

\chapter{Hudební teorie}
% základy hudební nauky
Hudební dílo tvoří několik částí, 
přičemž různé hudební žánry kladou důraz na jiný prvek.
Prvním z nich je melodie, uspořádaná sekvence jednotlivých tónů tak,
aby šly zahrát, nebo zazpívat. 
Melodie vyjadřuje hlavní hudební myšlenku skladby. 
Přidáním více souznějících tónů dostáváme harmonii -- vertikální složku hudby.
Rytmus dodává hudbě dynamiku střídáním dlouhých a krátkých,
přízvučných a nepřízvučných tónů.
\par

Nejen pro harmonizaci melodie je potřeba znát hudební nauku. 
V této kapitole budou nastíněny základy a pojmy hudební nauky, 
tedy co to jsou tóny, noty, jejich vlastnosti a stupnice. 
Dále bude rozdělena melodie a harmonie a popsány akordy a tóniny. 
\par 

\section{Tóny}
Chvěním těles, které rozechvívají okolní vzduch, vzniká zvuk, 
tedy to co slyšíme. 
Nepravidelným chvěním vznikají hluky. 
Naopak zvuky vznikající pravidelným kmitáním tělesa nazýváme tóny. 
V hudbě jsou především využívány právě tóny. 
Ty mají čtyři základní vlastnosti. 
\par

Podle různé doby chvění pružného tělesa rozlišujeme délku tónu. 
Další vlastností je síla, která je dána vzdáleností krajních bodů,
mezi nimiž těleso kmitá.
Tato vzdálenost se také označuje rozkmit nebo amplituda.
Velikost amplitudy je přímo úměrná síle, hlasitosti tónu.
Podle původu rozlišujeme barvu, někdy také témbr tónu.
Barva závisí na počtu znějících alikvotních tónů,
které se rozezní současně s hlavním hraným tónem. 
To je závislé na materiálu chvějícího se tělesa.
Pro představu, lze rozeznat, zda zní klavír, housle, trumpeta, či mužský nebo ženský hlas, 
navzdory tomu, že všechny mají stejnou výšku. 
Rozmanitou barvitost mají především velké orchestry.
Výšku tónu určujeme podle frekvence chvění, čili počtu kmitů za vteřinu. 
Čím vyšší frekvence, tím vyšší tón, a čím nižší frekvence, tím hlubší tón.
\cite{zenkl,cmiral} 
\par

\section{Soustava a jména tónů}
Tónová soustava je přehledné uspořádání všech tónů užívaných v hudbě.
Z vlastností tónů, vyjmenovaných výše, tónová soustava bere v úvahu pouze výšky tónu. 
Současnou tónovou soustavu tvoří sedm základních tónů, které
byly v minulosti označeny písmeny obyčejné abecedy: a, b, c, d, e, f, g.
Později se jako počátek hudební abecedy určilo písmeno c 
a písmeno b se nahradilo písmenem h.
Tím vzniká nynější hudební abeceda c, d, e, f, g, a, h.
\par

Těchto sedm tónů se pravidelně opakují ve vyšších a nižších polohách.
Při postupu od výchozího c k nejbližšímu následujícímu, respektive předchozímu, c nacházíme osm stupňů 
(včetně obou c).
Vzdálenost mezi dvěma tóny stejného jména se proto nazývá oktáva (z latinského octo - osm).
Tónová soustava má oktáv devět.
Jelikož se základní tóny opakují stejnými jmény v různé výši,
jsou tyto oktávy dále pojmenovány.
Díky tomu má každý jednotlivý tón nejen svůj vlastní název,
ale také ustálené označení (viz tabulka \ref{tabulkaOktav}).
Například tón e v dvoučárkové oktávě se nazývá dvoučárkové e
a značí se buď $e^2$, nebo{ e''}.
\cite{zenkl,cmiral}
\par

\begin{table}[]
    \begin{tabular}{ l l l l l l l | l }
        $C_2$ & $D_2$ & $E_2$ & $F_2$ & $G_2$ & $A_2$ & $H_2$ & Subkontra oktáva    \\
        $C_1$ & $D_1$ & $E_1$ & $F_1$ & $G_1$ & $A_1$ & $H_1$ & Kontra oktáva       \\
        C     & D     & E     & F     & G     & A     & H     & Velká oktáva        \\
        c     & d     & e     & f     & g     & a     & h     & Malá oktáva         \\
        $c^1$ & $d^1$ & $e^1$ & $f^1$ & $g^1$ & $a^1$ & $h^1$ & Jednočárková oktáva \\
        $c^2$ & $d^2$ & $e^2$ & $f^2$ & $g^2$ & $a^2$ & $h^2$ & Dvoučárková oktáva  \\
        $c^3$ & $d^3$ & $e^3$ & $f^3$ & $g^3$ & $a^3$ & $h^3$ & Tříčárková oktáva   \\
        $c^4$ & $d^4$ & $e^4$ & $f^4$ & $g^4$ & $a^4$ & $h^4$ & Čtyřčárková oktáva  \\
    \end{tabular}
    \caption{Značení tónů a názvy jednotlivých oktáv}
    \label{tabulkaOktav}
\end{table}

Kromě označení tónů písmeny se lze také setkat se solmizačními slabikami do, re, mi, fa, sol, la, si,
které v 10. století zavedl italský mnich Guido z Arezza.
\cite{cmiral}
\par    

\section{Alterace}
Každý ze základních tónů tónové soustavy můžeme jednou nebo dvakrát snížit nebo zvýšit.
Takto zvýšené a snížené tóny se souhrnně nazývají odvozené nebo alternované.
Alterace je pak shrnující název pro zvyšování a snižování tónů.
Zvýšení tónu značíme příponou -is v jeho názvu.
Naopak snížení označujeme příponou -es.
Existují ovšem různé výjimky.
Například snížením tónů~e~a~a získáme es, respektive as.
Pro snížené h se, místo hes, používá název~b.
\cite{zenkl}
\par

Alterace posunuje tón vždy právě o půltón, 
což je nejmenší vzdálenost mezi dvěma tóny užívaná v naší hudbě.
Celý tón je pak tvořen dvěma půltóny. 
V rozmezí oktávy se nachází dva půltóny ({e-f, h-c}), a pět celých tónů.
Dohromady oktáva sestává z dvanácti půltónů.
Přehledné uspořádání tónů a půltónů v oktávě lze vidět na klaviaturách klávesových nástrojů (viz obrázek \ref{obrazekRozlozeniKlaviatury}).
Základní tóny leží na bílých klávesách, jejich alterace pak na černých.
\cite{zenkl,cmiral}
\par

Můžeme si všimnout tónů, které mají stejnou výšku, ale různá jména.
Těmto tónům říkáme enharmonické, enharmonická záměna je pak nahrazení tónu tónem stejné výšky, ale jiného jména.\cite{zenkl}

\begin{figure*}[h]\centering
    \centering
    \includegraphics[width=0.4\linewidth]{obrazky/klaviatura.jpg}\\[1pt]  
    \caption{Přehledné uspořádání tónů a půltónů v oktávě \cite{orientaceNaKlaviature}}    
    \label{obrazekRozlozeniKlaviatury}
\end{figure*}

\section{Rytmus}
Melodii hudebních skladeb se málokdy vyskytují tóny a noty pouze jediného druhu.
Zpravidla se střídají dlouhé a krátké tóny, přízvučné a nepřízvučné.
Kolem přízvučného (silněji hraného) tónu se vyskytuje skupina delších a kratších tónů.
Tyto tvoří rytmický útvar a opakování rytmických útvarů udává rytmus.
\todo{dopsat úvahu? z cmírala str38}
\par
Rytmus je patrný především v hudbě určené pro tanec.
Mezi oblíbené tance s nápadnými rytmy patří například česká polka,
valčík a moderní tango. 
Některé tance a rytmy jsou spíše regionálního charakteru, 
mezi ně řadíme polské tance polonéza a mazurka, maďarský čardáš, či španělské bolero.
Složitými rytmy pak vynikají písně moravské a slovenské.
Rytmus je podle některých
\footnote{\"Na počátku byl rytmus\" -- Hans Bülow} 
nejpodstatnější složkou hudby.
To je patrné především v hudbě národů na nízkém stupni kultury,
kde rytmy určené bicími nástroji, doplněné tleskáním a podupáváním,
zaujímají přední postavení tamní hudby.
\cite{cmiral}

\section{Stupnice a tónina}

Stupnice je postupná řada tónů v rozsahu jedné oktávy.
Při poslechu je zřejmé, že jednotlivé stupně mezi sebou nemají konstantní vzdálenost.
\cite{kofron}
Některé dvojice dělí půltón, čili malá sekunda, 
jiné větší krok, celý tón, velká sekunda.
Ta je rovna vzdálenosti dvou půltónů.
Vzdálenosti jednotlivých stupňů jsou dány pravidly.
\cite{cmiral}
Stupnice je možné rozdělit podle vzdáleností do několika kategorií.

\begin{enumerate}
	\item diatonické
	\begin{enumerate}
        \item staré (církevní)
        \item moderní (durové, mollové)
    \end{enumerate}
	\item chromatické
	\begin{enumerate}
        \item chromatické
        \item alterované
    \end{enumerate}
	\item exotické
	\begin{enumerate}
        \item cikánská
        \item pentatonická (čínská)
        \item celotónová
    \end{enumerate}
\end{enumerate}

V diatonických stupnicích jsou od sebe stupně vzdáleny jak o celé tóny, tak o půltóny.
V chromatických pak většinou pouze o půltóny.
Exotické stupnice se od předchozích liší zvláštním uspořádáním stupňů.
\cite{kofron}
\par

Stupnici lze vytvořit od libovolného tónu.
V praxi se však používají jen ty, 
v nichž se uplatňují odvozené tóny nanejvýš dvakrát zvýšené, nebo snížené.
Výchozí tón stupnice je jejím základním tónem, a podle něj se stupnice jmenuje.
Například stupnice G dur začíná tónem g, fis moll tónem fis.
Rozdělení dur nebo moll se určuje podle vzdáleností mezi jednotlivými stupni.
\cite{zenkl}
Durové stupnice označujeme velkým písmenem, mollové malým.
Jednotlivé stupně stupnice se značí římskými číslicemi.
Mimo to má každý stupeň svůj funkční název:
\cite{kofron}

\begin{enumerate} [label=\Roman*]
    \item tónika
    \item supertónika (super = nad) nebo střídavá dominanta
    \item vrchní medianta
    \item subdominanta (spodní dominanta)
    \item dominanta (dominující, vládnoucí tón)
    \item spodní medianta
    \item citlivý tón - tíhne k půltónovému pokračování k tónice
\end{enumerate}
\par

Tóniny a stupnice mají stejné základní tóny, a od toho také jména.
Opět, například pokud je tónina G dur, pak jejím základním tónem, tónikou je tón g.
Tónina je volné pořadí tónů stupnice tvořící hudbu.
Není nutné aby byly použity všechny tóny stupnice, 
ale jejich uspořádání musí dát vyniknout tónice.
Tónika se určuje sluchem pomocí tonálního cítění.
Velmi často melodie tónikou začíná, nebo se k ní vrací.
V závěru melodie bývá tónika uplatňována nejčastěji.
V takovém případě působí melodie přesvědčivě uzavřena 
a posluchač necítí potřebu dalšího pokračování.
\cite{zenkl}
\par

\subsection{Durové stupnice}
Základní durovou stupnicí je C dur (c, d, e, f, g, a, h, c).
Durové, neboli tvrdé stupnice jsou charakteristické III. stupněm,
který s I. stupněm tvoří interval velké tercie.
Jinými slovy, první a třetí tón stupnice jsou vzdáleny dva celé tóny.
Vzdálenost mezi jednotlivými stupni durové stupnice jsou dva celé tóny, půltón, tři celé tóny, půltón.
$ 1 - 1 - 1/2 - 1 - 1 - 1 - 1/2 $
Půltóny jsou tedy mezi III.--IV. a VII.VIII. stupněm


\section{Notopis}
%   noty (délky, výšky)
Alfabetické znaky nejsou pro značení tónů dostatečné, protože popisují pouze výšku tónu.
Pro hru na hudební nástroj, nebo skládání hudby je potřeba specifikovat také zbývající vlastnosti.
Po staletí trvajícím vývoji se ustálil a používá se notopis.\par

\subsection*{Notová osnova}
Základem jsou noty a notová osnova, kterou tvoří pět linek a čtyři mezery.
Noty se zapisují na linky i do mezer.
Takto lze zapsat pouze devět not.
Notovou osnovu však můžeme rozšířit pomocnými linkami nad, popřípadě pod notovou osnovou. 
Pomocné linky zapisujeme pouze tak dlouhé, jak je nezbytně nutné.
Pro snadnou orientaci v osnově se v praxi počítají linky a mezery směrem nahoru (s výjimkou pomocných linek pod osnovou). 
Linky a mezery se počítají zvlášť. 
První mezera je mezi první a druhou linkou.\par

\todo{ TODO Obrazek notove osnovy} 

\subsection*{Noty}
Samotné noty jsou písemné značky pro tóny.
Tvar noty vyjadřuje délku noty a pozice v notové osnově její výšku.
Nota se skládá z několika částí.
Hlavička určuje pozici noty, a může být buď vyplněná , nebo nevyplněná . 
Nožka opět buď může být, a nebo nemusí . 
Pokud nota nožku má, pak ji, u not na třetí, prostřední lince a vyšších, píšeme směrem dolů ,
a naopak u nižších not směrem nahoru.
I toto pravidlo má ovšem nějaké výjimky.
Poslední částí noty je praporec, těch může mít nota nula až čtyři .
Pokud je za sebou víc not s praporci, lze je spojit, a praporce nahradit trámci.
V hudbě se používají tyto základní noty:

\todo{ TODO obrazek vsech not a mozna pomlk}

%\begin{itemize}
%    \item \wholeNote celá
%    \item \halfNote \halfNoteDown půlová
%    \item \quarterNote \quarterNoteDown čtvrťová
%    \item \eighthNote \eighthNoteDown osminová
%    \item \sixteenthNote \sixteenthNote šestnáctinová
%    \item \thirtysecondNote \thirtysecondNoteDown dvaatřicetinová
%\end{itemize}

Podle názvů not je zřejmé, že nota celá má délku dvou půlových not, 
čtyř čtvrtinových osmi osminových a tak dále.
Nejbližší nižší nota má délku poloviční a nejbližší vyšší dvojnásobnou.
\par

\subsection*{Klíče}
Jména (výšky) not v osnově jsou určeny takzvaným klíčem, 
který se píše vždy na začátek každého řádku.
Klíčem se označuje pozici a jméno jedné notě a od ní se pojmenují ostatní v pořadí, 
přičemž pojmenování se řídí hudební abecedou.
Tvary klíčů se vyvinuli z písmen not, jejichž linku označují.
V současné dob se používají čtyři klíče(viz obrázek \ref{obrazek4Klice}), 
a to G-klíč (houslový), F-klíč (basový) a dva C-klíče (altový a tenorový).

\todo{obrazek 4 Klice}

Houslový klíč označuje notu $g^1$ na druhé lince.
Basový určuje malé f na čtvrté lince, altový $c^1$ na třetí a tenorový také $c^1$ na čtvrté lince.
Basový klíč je nejvhodnější pro notaci hlubokých tónů a houslový k notaci vysokých.
Tyto dva klíče se používají nejčastěji. 
Pro orientaci v těchto klíčích je dobré vědět, že
tón $c^1$ se v houslovém klíči nachází na první spodní pomocné lince, 
a v basovém se nachází na první horní pomocné lince.
\cite{cmiral,zenkl}
\par

\subsection*{Pomlky}
Kromě značení tónů je v hudbě potřeba zaznačit také odmlky, tedy místa kdy má být ticho.
Ticho se značí zvláštními značkami, pomlkami v notovém zápise a mají stejné hodnoty jako noty.
Délka pomlky je opět, jako u not, dána tvarem (viz obrázek \ref{obrazekPomlky}).
Celá pomlka značí pomlku na jakýkoliv celý takt.
\cite{cmiral,zenkl}

\todo{obrazek Pomlky}

\subsection*{Posuvky}
Samotné noty značí pouze základní tóny.
Pokud chceme, aby nota značila zvýšený tón, je potřeba před ni (na její linku, nebo mezeru) napsat křížek ($\sharp $).
Je-li před notou béčko ($\flat $), pak nota značí tón snížený.
Posuvky lze psát puď přímo před notu, kterou chceme alterovat, nebo na začátek každého řádku, pak předznamenává tóninu.
Platnost posuvky před notou je do konce taktu.
Je možné ukončit účinnost dříve, odrážkou ($\natural $).
Zde ovšem platí, že pokud byla posuvka na začátku řádku, odrážka platí do konce taktu.
\cite{cmiral,zenkl}

\subsection*{Takty}
Noty a pomlky se řadí do krátkých úseků, ze kterých se skládá celá hudební skladba.
Tyto úseky nazýváme takty.
V notové osnově jsou od sebe odděleny tenkými svislými čarami.
Pokud je skladba pro více nástrojů a je notována na více řádků. pak protíná taktová čára všechny řádky.
Na konci skladby je taktová čára doplněna ještě jednou silnější. 
Takty dělíme na doby -- stejně dlouhé časové úseky.
Podle počtu dob jsou takty dvoudobé až třídobé.
Doby mohou být půlové, osminové, nejčastěji však čtvrťové.
První doba je vždy přízvučná, hraje se silněji.
Ostatní doby jsou hrány bez přízvuku, pokud nejsou opatřeny zvláštní značkou~($>$) nad notou.
Takto popsané jsou takty jednoduché.
Spojením dvou jednoduchých taktů vzniká takt složený.
Složené takty mají dva přízvuky.
Silnější na začátku taktu, a o něco slabší tam, kde by začínal druhý jednoduchý takt.\par
Součet hodnot not v každém jednom taktu se musí rovnat počtu dob taktu.
Čili například v jednom  dvoučtvrťovém taktu může být jedna nota půlová (má dvě doby), dvě noty čtvrťové, jedna čtvrťová a dvě osminové, a tak dále.
To, jaký má skladba takt, zjistíme na začátku skladby, kde je zapsán zlomek.
Čitatel uvádí počet dob, a jmenovatel hodnotu počítané doby.
V průběhu skladby může dojít ke změně taktu.
Tuto změnu značíme taktovým označením hned za taktovou čáru, kde změna začíná.
Čtyřčtvrťový takt se většinou nazývá celý, a místo zlomku je značen velkým písmenem C.
\cite{cmiral,zenkl}

% melodie / harmonie
% akordy (?)

\chapter{Reprezentace dat}
% ABC, MIDI, GUIDO ... viz NOTES.md
V této části je popsán vstupní formát trénovacích dat.

\section{MIDI soubory}
MIDI (Musical Instrument Digital Interface) je volně přístupný standard,
který specifikuje hardware i software pro digitální komunikaci hudebních nástrojů,
sekvencerů, počítačů, mixérů, a dokonce i jevištní techniky, jako jsou reflektory a lasery.
\cite{MIDI_tutorials}
\par

Standardní MIDI soubor (zkráceně SFM) je formát binárních souborů 
určených pro uložení a přenos hudebních dat mezi zařízeními.
Soubory v tomto formátu lze rozeznat pomocí \emph{.mid} přípony.
Oproti ostatním formátům pro uložení zvuku SFM neukládají digitalizovaný zvuk,
ale parametry použitých nástrojů, informace o tempu, kanálech, 
jednotlivé noty uložené jako kombinace (čas, hodnota, rychlost) a další MIDI události.
Data z těchto souborů jsou načteny do nějakého přehrávače 
a výsledný zvuk je vytvořen připojeným sound-enginem.
\cite{MIDI_tutorials,Neznamy_aboutMIDIFiles}
\par
Data v SFM jsou uložena v blocích.
Každý z těchto bloků začíná čtyřbajtovým textovým označením 
a čtyřbajtovou délkou datového bloku udávající počet bajtů.
Ihned po délce začíná datová část o délce zadané v hlavičce.
Za datovou částí začíná buď další blok, nebo konec souboru.
\cite{MIDI_tutorials,Back_SMF_Specif}
\par
Soubory MIDI mají dva druhy bloků. 
Hlavičkový blok poskytuje informace vztahující se k celému souboru.
Bloky stopy obsahují proudy MIDI dat až šestnácti MIDI kanálů.
MIDI soubor vždy začíná hlavičkovým blokem 
a je následován jedním, nebo více bloky stop.\cite{Back_SMF_Specif}

\begin{itemize} [label={}]
    \item MThd <délka hlavičky>
    \item <data hlavičky>
    \item MTrk <délka stopy>
    \item <data stopy>
    \item MTrk <délka stopy>
    \item <data stopy>
    \item \dots
\end{itemize}

\subsection{Hlavička}
Blokem hlavičky začíná každý MIDI soubor a obsahuje jeho základní popis.
Je uvozen typem bloku, čtyřmi ASCII znaky "MThd" 
a 32bitovou reprezentací čísla 6 (délka bloku). 
Začátek SMF je tedy pevně daný, 
a takový soubor lze na první pohled rozeznat začínající sekvencí 
\emph{0x4D 54 68 64 00 00 00 06}. 
Datovou část tvoří tři 16bitové slova.
První z nich specifikuje formát, celkovou organizaci souboru.
Tím je myšleno uspořádání uložených stop.
Jsou zde tři možnosti:

\begin{enumerate}\addtocounter{enumi}{-1}
    \item soubor obsahuje jedinou více-kanálovou stopu
    \item několik souběžných stop
    \item několik nezávislých stop
\end{enumerate}

Formát 0 je nejrozšířenější a je podporován i nejjednoduššími programy a hardware.
\cite{Back_SMF_Specif}
Formát 1 obsahuje jednu nebo více vertikálně synchronních stop.
Jinými slovy, při přehrávání takového souboru začnou všechny stopy ve stejný čas 
a mohou reprezentovat různé části písničky.
V posledním, asynchronním formátu nemusí stopy nutně začínat ve stejnou dobu
\cite{Neznamy_aboutMIDIFiles}.
Tento formát je nejméně používaný
\cite{MIDI_tutorials}.
\par

Dalším slovem hlavičky je počet stop.
Ten je u nultého formátu vždy roven jedné.
Poslední část definuje kódování času, 
které může být zadáno dvěma způsoby.
Ty jsou dány prvním bitem slova.
Pokud je bit nulový, pak zbývající bity reprezentují počet "tiků" generátoru hodin, 
které se vlezou do jedné čtvrťové noty.
Pokud je bit nastaven na 1, pak je čas vyjádřen jako rozdělení vteřin podle SMPTE standardu
a MIDI Time Codu.
\cite{Back_SMF_Specif}
Filmovým průmyslem vytvořený SMPTE, mimo jiné, definuje čtyři různé snímkovací frekvence.
Je podle něj možné rozdělit vteřinu na 24, 25, 29 nebo 30 snímků.
Pro hudební účely je však potřeba ještě jemnější rozlišení.
Každý snímek je tedy dále možné rozdělit na "sub-snímky".
\cite{Neznamy_aboutMIDIFiles}
Bity 14 až 8 časového kódování udávají počet SMPTE snímků za vteřinu 
a zbývající část je rozlišení jednoho snímku.
Pro představu, při časování událostí na milisekundy je potřeba nastavit 25 SMPTE snímků,
a rozlišení 40, neboť $ 40 * 25 = 1000 $ "subsnímků" za vteřinu, stejně jako milisekund. 
\cite{Back_SMF_Specif}
\par

\subsection{Stopy}
Každý MIDI soubor obsahuje nejméně jeden blok stopy.
Ten obsahuje parametry použitých nástrojů, jednotlivé noty skladby, 
ale také různé textové informace jako název skladby nebo její text.
Za identifikátorem bloku "MTrk" a jeho délkou následuje posloupnost MIDI událostí
ve tvaru časový přírustek, stavový bajt, datové bajty.
\cite{Back_SMF_Specif}
\par
Časový přírustek (deltatime) značí počet "tiků" od předchozí události.
Tato hodnota má proměnnou délku.
Tento formát umožňuje velkým číslům využít tolik bajtů, kolik potřebují bez toho,
aby malá čísla plývala místem a plnila jej nulami.
Hodnoty jsou kódovány do 7-bitových bajtů 
a osmý, nejvýznamnější bit je nastaven na jedničku.
Pokud je bajt poslední v posloupnosti, pak na nulu.
Celý deltatime by však ideálně neměl přesáhnout 4 bajty.
\cite{Neznamy_aboutMIDIFiles}
První událost stopy, nebo dvě současné události mají časový přírustek roven nule.
\cite{Back_SMF_Specif}
\par



\chapter{Přehled přístupů automatické harmonizace}
Pojem automatická harmonizace melodie označuje disciplínu, která se zabývá tvorbou modelu, 
který dokáže vygenerovat harmonický doprovod k dané melodii.
Samotná harmonizace je složitý úkol, 
protože tu samou melodii můžeme harmonizovat mnoha různými způsoby.
Správná harmonie závisí na subjektivním pocitu, hudebním žánru a dalších faktorech.
V hudbě se ustálilo mnoho pouček a zákonů o tom, které akordy k sobě patří,
a jak na sebe navazují.
Často určují jemné nuance, či jsou podmíněny kulturním vlivem, 
což je pro stroj velmi obtížné zachytit.
\cite{YinCheng_comparativeStudy}
\par
V této kapitole bude představeno několik přístupů právě k automatické harmonizaci.

\section{Model založený na shodě šablon}
\label{shodaSablon}
Tento model nejdříve rozdělí všechny melodie z trénovací sady po polovinách taktů, 
a pro každý, takto vytvořený segment vytvoří profil výšky tónů 
(dále jen PCP\footnote{z anglického Pitch Class Profile}).
\cite{YinCheng_comparativeStudy}
PCP je dvanácti-dimenzionální binární vektor, 
kde každá složka reprezentuje přítomnost každého z dvanácti možných půltónů v daném půl-taktu.
\cite{fujishima}
PCP nového segmentu se porovná s těmi z trénovací sady
a spočítá se podobnostní skóre.
Nový segment je pak označen takovým akordem,
který měl jemu nejpodobnější segment v trénovací sadě -- s nejvyšším skóre.
Jestliže má nejvyšší skóre několik segmentů,
je výsledný akord vybrán náhodně s rovnoměrným rozložením pravděpodobnosti.
Označení akordu je opět ve formě PCP, a to tak,
že všechny tóny náležící akordu jsou nastaveny do 1.
\par

Nevýhodou tohoto modelu je, že určuje jednotlivé segmenty nezávisle bez toho,
aby bral v úvahu sousední akordy, nebo postup akordů ve skladbě.
\cite{YinCheng_comparativeStudy}

\section{Skryté Markovovy modely}
Jednou z nejpoužívanějších metod generování akordů a harmonizace melodie, 
před současným rozšířením hlubokého učení, byly skryté Markovovy modely 
(dále HMM \footnote{z anglického Hidden Markov Model}).
HMM jsou pravděpodobnostní nástroj s jehož pomocí lze modelovat sekvence se skrytými proměnnými.
Označení akordů je považováno právě za skrytou proměnnou a 
HMM se snaží odhadnout nejpravděpodobnější sekvenci akordů na k daným notám melodie.
Na rozdíl od předchozího přístupu tento model bere v úvahu předcházející akordy.
\cite{YinCheng_comparativeStudy}

\chapter{Strojové učení}
% rešerše strojového učení, jak funguje
Umělá inteligence dokázala relativně rychle řešit problémy, 
které jsou pro člověka intelektuálně náročné, 
ale jsou popsány formálními pravidly.
Avšak pro člověka intuitivní \"problémy\" lze jen s těží formálně popsat.
Strojové učení umožňuje počítači naučit se ze zkušeností
a porozumět světu v souvislostech.
Získáváním zkušeností je odbourán proces formálního zápisu veškerých vědomostí,
které jsou k řešení problému potřeba.
\cite{Goodfellow-et-al-2016}
Harmonizace melodie je někde na půl cesty mezi těmito extrémy.
Na jednu stranu popsána pravidly hudební nauky,
na stranu druhou nejsou pravidla zcela striktní 
a ponechávají skladateli volnou ruku.
\par

Strojové učení je v podstatě forma aplikované statistiky,
s důrazem na odhad složitých funkcí počítačem.
Většina algoritmů lze rozdělit do dvou kategorií -- učení s učitelem (supervised) 
a učení bez učitele (unsupervised).
Při učení s učitelem se algoritmus snaží nějakému vstupu přiřadit daný výstup
na základě trénovací sady příkladů vstupů a výstupů.
Trénovací výstupy může být obtížné získat automaticky,
a proto musí být dodány člověkem, učitelem.
Mezi algoritmy učící se s učitelem patří například Support Vector Machines,
k-nejbližších sousedů, nebo rozhodovací stromy.
\cite{Goodfellow-et-al-2016}
\par

Algoritmy bez učitele se učí pouze na základě vlastností objektů 
a už nemají reflexi, zda se naučili správně, nebo ne.
Učení bez učitele se zabývá extrakcí informací z rozdělení,
u kterého nejsou potřeba člověkem označené trénovací data.
Nejčastěji se jedná o odhad hustoty, odšumění dat s nějakým rozdělením,
hledání střední hodnoty všech dat nebo shlukování dat do skupin podobných objektů.
Hledáme tedy nejlepší reprezentaci dat.
Tato reprezentace data zjednodušuje, ale tak, aby bylo zachováno co nejvíc informací.
Mezi představitele tohoto přístupu patří například shlukování k-nejbližších sousedů, 
analýza hlavních komponent nebo lineární regrese. 
\cite{Goodfellow-et-al-2016}

\section{Neuronové sítě}
Neuronové sítě jsou množina algoritmů určených k rozpoznávání vzorů.
Vzory rozumíme jakékoliv data, čísla, obrázky, zvuk, text a další.
Ty musí být převedeny na vektory čísel, 
které jsou následně použity jako vstup.
Tyto sítě se také nazývají dopředné (Feedforward Neural Networks).
\cite{Nicholson_NeuralNets}
\par

Neuronové sítě mají většinou několik propojených vrstev.
První vrstva se také nazývá vstupní, 
zatímco poslední je výstupní.
Vrstvy mezi vstupní a výstupní jsou takzvaně skryté.
Počet vrstev udává hloubku modelu, odtud také hluboké učení, 
a velikost vektoru jeho šířku.

\begin{figure*}[h]\centering
    \centering
    \includegraphics[width=0.6\linewidth]{obrazky/NNSchema.png}\\[1pt]  
    \caption{Schéma neuronové sítě}    
    \label{obrazekSchemaNeuronoveSite}
\end{figure*}

Úkolem neuronové sítě je aproximovat neznámou funkci $ y=f'(x)$
funkcí $ y=f(x;\theta )$, kde \emph{x} a \emph{y} značí vstup, respektive výstup.
K hodnotě $\theta$ dojde neuronová síť tak,
aby výsledný model nejlépe aproximoval původní funkci.
Samotná funkce \emph{f} se skládá z několika zřetězených funkcí 
$f(x)=f^{(3)}(f^{(2)}(f^{(1)}(x)))$,
kde $f^{(n)}$ označuje vrstvy sítě, jejichž vstupem je vektor z předchozí vrstvy 
a výstupem je opět vektor.
\cite{Goodfellow-et-al-2016}

\subsection*{Neurony}
Mnohem obvykleji jsou vrstvy reprezentovány množinou uzlů, které nazýváme neurony.
Vstupem každého neuronu je celý vektor předchozí vrstvy 
a výstupem je skalární hodnota, prvek výstupního vektoru jeho vrstvy.
Paralelním výpočtem všech neuronů vrstvy je získán její výsledek.
\cite{Goodfellow-et-al-2016}
\par

Každému vstupu neuronu je přiřazena váha, která jej buď posílí nebo utlumí.
Suma všech váhovaných vstupů poté prochází takzvanou aktivační funkcí,
která určuje jestli, a jak moc může signál pokračovat skrz síť.
Pokud signál projde, pak o neuronu můžeme říct, že byl aktivován.
\cite{Nicholson_NeuralNets}

\begin{figure*}[h]\centering
    \centering
    \includegraphics[width=0.6\linewidth]{obrazky/neuronSchema.png}\\[1pt]  
    \caption{Schéma neuronu}    
    \label{obrazekSchemaNeuronoveSite}
\end{figure*}

\todo{TODO - Dopsat princip uceni NN}

\section{Rekurentní neuronové sítě}
Samotné neuronové sítě jsou pro některé typy problémů pořád příliš omezené.
Jejich vstupem je vektor pevně dané velikosti, 
výpočet probíhá v předem známém počtu kroků a na jeho konci je opět vektor pevné délky.
Taková omezení neberou v úvahu proudová data, tedy sekvence na sobě závislých vektorů.
Rekurentní sítě umožňují mít takovou sekvenci jako vstup, jako výstup, nebo jako obojí.
\cite{Karpathy_RNNs}.

\begin{figure*}[h]\centering
    \centering
    \includegraphics[width=0.6\linewidth]{obrazky/RNNSchema.png}\\[1pt]  
    \caption{Schéma rekurentní neuronové sítě\cite{colah_lstm}}    
    \label{obrazekSchemaRekurentniNeuronoveSite}
\end{figure*}

\section{LSTM}
\label{lstmTeorie}
Long Short-Term Network (dále jen LSTM), jsou speciálním druhem rekurentních neuronových sítí.
Jsou schopny se naučit i dlouhodobější závislosti v sekvencích dat.
S touto vlastností byly primárně navrženy.\cite{colah_lstm}
\par
Narozdíl od RNN mají LSTM čtyři vrstvy v jednom opakujícím se modulu.

% frameworky ... viz NOTES.md

\chapter{Návrh systému}
V této kapitole je představen návrh systému,
který umožní automatické generování harmonie na základě známé melodie.
Jak bylo napsáno, harmonie je souzvuk více zvuků najednou.
Aby systém dokázal generovat tóny, které spolu opravdu souzní,
musí obsahovat správně naučený model.
Ten bude pro každý půltakt generovat akord,
který s ním ladí.

\section{Vstup}
Vstupem celého systému je MIDI soubor, obsahující melodii,
pro kterou chceme vygenerovat harmonický doprovod.
Neuronové sítě si ale s takovými soubory neumí moc poradit,
a tak je soubor potřeba převést na vhodnější reprezentaci.
Tou jsou vektory.
Jak melodii převést na vektor bylo nastíněno v části \ref{shodaSablon}.
Vstup bude tedy rozdělen po polovinách taktů, 
a pro každý se spočítá jeho PCP -- dvanácti-dimenzionální vektor,
který obsahuje jaké tóny a jak dlouho v daném půltaktu zní.
Výsledná posloupnost vektorů bude vstupem modelu.

\section{Model}
Základem modelu strojového učení bude LSTM rekurentní neuronová síť.
Díky nim se bude model učit nejenom na základě aktuálního vstupu,
ale také bude brát v úvahu parametry předchozích stavů.
Samotná rekurentní neuronová síť však tyto parametry nedrží dostatečně dlouho.
Proto budou využity LSTM vrstvy, 
které tuto limitaci vylepší.
Jak již bylo naznačeno, 
vstupní vektor má 12 dimenzí.
Tolik tedy bude vstupních uzlů.
Počet vnitřních vrstev bude empiricky vybrán na základě experimentů.
\par
Při učení modelu bude potřeba vyhodnotit jak moc se mýlil.
Toho dosáhneme tak, že výslednou sekvenci porovnáme s
testovací a porovnáme, v kolika případech se model spletl.

\section{Výstup}
Výstupem modelu bude stejný vektor jako na vstupu.
Nyní ale bude PCP obsahovat jen tóny akordu, 
který má znít v daném půltaktu.
Pokud budeme brát v úvahu pouze kvintakordy (dur, moll, zvětšený, zmenšený),
pak může mít výstup 48 možných stavů.
Vezmeme-li v úvahu také možnost, že v daném půltaktu nehraje žádný akord,
existuje 49 tříd.
\par
Výsledná sekvence akordů bude přidána k původní melodii
a vytvoří se nový, harmonizovaný MIDI soubor.


% popis systému, který bude výstupem práce

% vstupy
% způsob práce
% výstup

\chapter{Implementace}
% popis implementace

\chapter{Experimenty}
% experimentování s implementovaným systémem
% představení výsledků

\chapter{Diskuse}
% diskuse výsledků
% jak je použít
% srovnání s cizími pracemi a papery

\chapter{Závěr}
\label{zaver}
% shrnutí práce
% možný budoucí vývoj práce
% celkový přínos
Cílem této práce bylo seznámit se s harmonizací známé melodie a s možností automatizace této činnosti za pomoci neuronových sítí.
Následně navrhnout systém, který bude automatickou harmonizaci umožňovat.
\par
Základům hudby byla věnována první kapitola, 
další se popisovala různé přístupy k automatické harmonizaci. Některé poznatky z těchto prací byly vybrány a použity v návrhu v následující kapitole.
Na základě tohoto návrhu bude poté, ve výsledné diplomové práci, implementován systém a jeho neuronová síť s využitím frameworku PyTorch.
Toto řešení bude, na základě testovacích dat, vyhodnoceno.
V závěru budou diskutovány výsledky a možné pokračování této práce.
