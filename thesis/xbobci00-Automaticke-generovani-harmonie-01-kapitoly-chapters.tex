\chapter{Úvod}
\label{uvod}
%úvod práce
%úvod do problému
%popis struktury práce


\chapter{Hudební teorie}
% základy hudební nauky
Pro harmonizaci melodie je potřeba znát hudební nauku. V této kapitole budou nastíněny základy a pojmy hudební nauky, tedy co to jsou tóny, noty, jejich vlastnosti a stupnice. Dále bude rozdělena melodie a harmonie a popsány akordy a tóniny. \par 
Chvěním těles, které rozechvívají okolní vzduch, vzniká zvuk, tedy to co slyšíme. Nepravidelným chvěním vznikají hluky. Naopak zvuk, vznikající pravidelným kmitáním tělesa, nazýváme tónem. V hudbě jsou především využívají právě tóny. Ty mají čtyři základní vlastnosti. Podle různé doby znění rozlišujeme délku tónu. Další vlastností je síla, která se v akustice nazývá hlasitost. Podle původu rozlišujeme barvu, někdy také témbr, tónu. Pro představu, lze rozeznat zda zní klavír, housle, trumpeta, či mužský nebo ženský hlas, ikdyž všechny mají stejnou výšku. \cite{abcZenkl}
%   noty (délky, výšky)
%   akordy (?)
% melodie / harmonie
%   tóniny a stupně

\chapter{Reprezentace dat}
% ABC, MIDI, GUIDO ... viz NOTES.md


\chapter{Přehled přístupů automatické harmonizace}
% viz 04_Automatic Melody Harmonization with Triad Chords - A Comparative Study.pdf


\chapter{Strojové učení}
% rešerše strojového učení, jak funguje
% neuronové sítě
% frameworky ... viz NOTES.md

\chapter{Návrh systému}
% popis systému, který bude výstupem práce
% vstupy
% způsob práce
% výstup

\chapter{Implmentace}
% popis implementace

\chapter{Experimenty}
% experimentování s implementovaným systémem
% představení výsledků

\chapter{Diskuse}
% diskuse výsledků
% jak je použít
% srovnání s cizími pracemi a papery

\chapter{Závěr}
\label{zaver}
% shrnutí práce
% možný budoucí vývoj práce
% celkový přínos
