\chapter{Úvod}
\label{uvod}
%úvod práce
%úvod do problému
%popis struktury práce


\chapter{Hudební teorie}
% základy hudební nauky
Nejen pro harmonizaci melodie je potřeba znát hudební nauku. 
V této kapitole budou nastíněny základy a pojmy hudební nauky, 
tedy co to jsou tóny, noty, jejich vlastnosti a stupnice. 
Dále bude rozdělena melodie a harmonie a popsány akordy a tóniny. \par 

Chvěním těles, které rozechvívají okolní vzduch, vzniká zvuk, 
tedy to co slyšíme. 
Nepravidelným chvěním vznikají hluky. 
Naopak zvuky vznikající pravidelným kmitáním tělesa, nazýváme tóny. 
V hudbě jsou především využívány právě tóny. 
Ty mají čtyři základní vlastnosti. 
Podle různé doby chvění pružného tělesa rozlišujeme délku tónu. 
Další vlastností je síla, která je dána vzdáleností krajních bodů,
mezi nimiž těleso kmitá.
Tato vzdálenost se také označuje rozkmit nebo amplituda.
Velikost amplitudy je přímo úměrná síle, hlasitosti, tónu.
Podle původu rozlišujeme barvu, někdy také témbr, tónu,
která zavisí na počtu znějících alikvotních tónů. 
Pro představu, lze rozeznat zda zní klavír, housle, trumpeta, či mužský nebo ženský hlas, 
navzdory tomu, že všechny mají stejnou výšku. 
Rozmanitou barvitost mají především velké orchestry.
Výšku tónu určujeme podle frekvence chvění, čili počtu kmitů za vteřinu. 
Čím vyšší frekvence, tím vyšší tón, a čím nižší frekvence, tím hlubší tón. 
\cite{zenkl}\cite{cmiral}
%   noty (délky, výšky)
%   akordy (?)
% melodie / harmonie
%   tóniny a stupně

\chapter{Reprezentace dat}
% ABC, MIDI, GUIDO ... viz NOTES.md


\chapter{Přehled přístupů automatické harmonizace}
% viz 04_Automatic Melody Harmonization with Triad Chords - A Comparative Study.pdf


\chapter{Strojové učení}
% rešerše strojového učení, jak funguje
% neuronové sítě
% frameworky ... viz NOTES.md

\chapter{Návrh systému}
% popis systému, který bude výstupem práce
% vstupy
% způsob práce
% výstup

\chapter{Implementace}
% popis implementace

\chapter{Experimenty}
% experimentování s implementovaným systémem
% představení výsledků

\chapter{Diskuse}
% diskuse výsledků
% jak je použít
% srovnání s cizími pracemi a papery

\chapter{Závěr}
\label{zaver}
% shrnutí práce
% možný budoucí vývoj práce
% celkový přínos
