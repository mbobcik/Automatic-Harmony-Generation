\chapter{Úvod}
\label{uvod}
%úvod práce
%úvod do problému
%popis struktury práce


\chapter{Hudební teorie}
% základy hudební nauky
Nejen pro harmonizaci melodie je potřeba znát hudební nauku. 
V této kapitole budou nastíněny základy a pojmy hudební nauky, 
tedy co to jsou tóny, noty, jejich vlastnosti a stupnice. 
Dále bude rozdělena melodie a harmonie a popsány akordy a tóniny. \par 

\section*{Tóny}
Chvěním těles, které rozechvívají okolní vzduch, vzniká zvuk, 
tedy to co slyšíme. 
Nepravidelným chvěním vznikají hluky. 
Naopak zvuky vznikající pravidelným kmitáním tělesa nazýváme tóny. 
V hudbě jsou především využívány právě tóny. 
Ty mají čtyři základní vlastnosti. \par

Podle různé doby chvění pružného tělesa rozlišujeme délku tónu. 
Další vlastností je síla, která je dána vzdáleností krajních bodů,
mezi nimiž těleso kmitá.
Tato vzdálenost se také označuje rozkmit nebo amplituda.
Velikost amplitudy je přímo úměrná síle, hlasitosti tónu.
Podle původu rozlišujeme barvu, někdy také témbr tónu.
Barva zavisí na počtu znějících alikvotních tónů,
které se rozezní současně s hlavním hraným tónem. 
To je závislé na materiálu chvějícího se tělesa.
Pro představu, lze rozeznat, zda zní klavír, housle, trumpeta, či mužský nebo ženský hlas, 
navzdory tomu, že všechny mají stejnou výšku. 
Rozmanitou barvitost mají především velké orchestry.
Výšku tónu určujeme podle frekvence chvění, čili počtu kmitů za vteřinu. 
Čím vyšší frekvence, tím vyšší tón, a čím nižší frekvence, tím hlubší tón.\cite{zenkl,cmiral} \par

\section*{Soustava a jména tónů}
Tónová soustava je přehledné uspořádání všech tónů užívaných v hudbě.
Z vlastností tónů, vyjmenovaných výše, tónová soustava bere v úvahu pouze výšky tónu. 
Současnou tónovou soustavu tvoří sedm základních tónů, které
byly v minulosti označeny písmeny obyčejné abecedy: a, b, c, d, e, f, g.
Později se jako počátek hudební abecedy určilo písmeno c 
a písmeno b se nahradilo písmenem h.
Tím vzniká nynější hudební abeceda c, d, e, f, g, a, h.\par
Těchto sedm tónů se pravidelně opakují ve vyšších a nižších polohách.
Při postupu od výchozího c k nejbližšímu následujícímu, respektive předchozímu, c nacházíme osm stupňů 
(včetně obou c).
Vzdálenost mezi dvěma tóny stejného jména se proto nazývá oktáva (z latinského octo - osm).
Tónová soustava má oktáv devět.
Jelikož se základní tóny opakují stejnými jmény v různé výši,
jsou tyto oktávy dále pojmenovány.
Díky tomu má každý jednotlivý tón nejen svůj vlastní název,
ale také ustálené označení (viz tabulka \ref{tabulkaOktav}).
Nakpříklad tón e v dvoučárkové oktávě se nazývá dvoučárkové e
a značí se buď $e^2$, nebo{ e''}.\cite{zenkl,cmiral}\par

\begin{table}[]
    \begin{tabular}{ l l l l l l l | l }
        $C_2$ & $D_2$ & $E_2$ & $F_2$ & $G_2$ & $A_2$ & $H_2$ & Subkontra oktáva    \\
        $C_1$ & $D_1$ & $E_1$ & $F_1$ & $G_1$ & $A_1$ & $H_1$ & Kontra oktáva       \\
        C     & D     & E     & F     & G     & A     & H     & Velká oktáva        \\
        c     & d     & e     & f     & g     & a     & h     & Malá oktáva         \\
        $c^1$ & $d^1$ & $e^1$ & $f^1$ & $g^1$ & $a^1$ & $h^1$ & Jednočárková oktáva \\
        $c^2$ & $d^2$ & $e^2$ & $f^2$ & $g^2$ & $a^2$ & $h^2$ & Dvoučárková oktáva  \\
        $c^3$ & $d^3$ & $e^3$ & $f^3$ & $g^3$ & $a^3$ & $h^3$ & Tříčárková oktáva   \\
        $c^4$ & $d^4$ & $e^4$ & $f^4$ & $g^4$ & $a^4$ & $h^4$ & Čtyřčárková oktáva  \\
    \end{tabular}
    \caption{Značení tónů a názvy jednotlivých oktáv}
    \label{tabulkaOktav}
\end{table}

Kromě označení tónů písmeny se lze také setkat se solmizačními slabikami do, re, mi, fa, sol, la, si,
které v 10. století zavedl italský mnich Guido z Arezza\cite{cmiral}.\par    

\section*{Alterace}
Každý ze základních tónů tónové soustavy můžeme jednou nebo dvakrát snížit nebo zvýšit.
Takto zvýšené a snížené tóny se souhrně nazývají odvozené nebo alternované.
Alterace je pak shrnující název pro zvyšování a snižování tónů.
Zvýšení tónu značíme příponou -is v jeho názvu.
Naopak snížení označujeme příponou -es.
Existují ovšem různé vyjímky.
Nakpříklad snížením tónů e a a získáme es, respektive as.
Pro snížené h se, místo hes, používá název b.\cite{zenkl}\par

Alterace posunuje tón vždy právě o půltón, 
což je nejmenší vzdálenost mezi dvěma tóny užívaná v naší hudbě.
Celý tón je pak tvořen dvěma půltóny. 
V rozmezí oktávy se nachází dva půltóny ({e-f, h-c}), a pět celých tónů.
Dohromady oktáva sestává z dvanácti půltónů.
Přehledné uspořádání tónů a půltónů v oktávě lze vidět na klaviaturách klávesových nástrojů (viz obrázek \ref{obrazekRozlozeniKlaviatury}).
Základní tóny leží na bílých klávesách, jejich alterace pak na černých.\cite{zenkl,cmiral}\par

Můžeme si všimnout tónů, které mají stejnou výšku, ale různá jména.
Těmto tónům říkáme enharmonické, enharmonická záměna je pak nahrazení tónu tónem stejné výšky, ale jiného jména.\cite{zenkl}

\begin{figure*}[ht]\centering
    \centering
    \includegraphics[width=0.4\linewidth]{obrazky/klaviatura.jpg}\\[1pt]  
    \caption{Přehledné uspořádání tónů a půltónů v oktávě \cite{orientaceNaKlaviature}}    
    \label{obrazekRozlozeniKlaviatury}
  \end{figure*}

%   noty (délky, výšky)
%   akordy (?)
% melodie / harmonie
%   tóniny a stupně

\chapter{Reprezentace dat}
% ABC, MIDI, GUIDO ... viz NOTES.md


\chapter{Přehled přístupů automatické harmonizace}
% viz 04_Automatic Melody Harmonization with Triad Chords - A Comparative Study.pdf


\chapter{Strojové učení}
% rešerše strojového učení, jak funguje
% neuronové sítě
% frameworky ... viz NOTES.md

\chapter{Návrh systému}
% popis systému, který bude výstupem práce
% vstupy
% způsob práce
% výstup

\chapter{Implementace}
% popis implementace

\chapter{Experimenty}
% experimentování s implementovaným systémem
% představení výsledků

\chapter{Diskuse}
% diskuse výsledků
% jak je použít
% srovnání s cizími pracemi a papery

\chapter{Závěr}
\label{zaver}
% shrnutí práce
% možný budoucí vývoj práce
% celkový přínos
